\section{Introducción}
\subsection{¿Qué es software libre?}
El software libre es aquel que respeta la libertad de los usuarios y de la comunidad, si se afirma de un programa de computadora que es software libre, es porque 
garantiza que se cumplan las cuatro libertades fundamentales que menciona la Free Software Foundation\cite{GNU}:
\begin{itemize}
 \item La libertad de ejecutar el programa para cualquier propósito.
 \item La libertad de estudiar cómo funciona el programa, y cambiarlo para que haga lo que usted quiera. 
 \item La libertad de redistribuir copias para ayudar a su prójimo.
 \item La libertad de distribuir copias de sus versiones modificadas a terceros. Esto le permite ofrecer a toda la comunidad la oportunidad de 
 beneficiarse de las modificaciones. El acceso al código fuente es una condición necesaria para ello. 
\end{itemize}
Por estas razones cuando se dice que un programa esta hecho en Java, $C\#$ o Python, no significa que sea software libre, si como usuario o cliente del software no
se garantiza el respeto integral de estas cuatro libertades. 
\subsection{¿Qué es un Erp?}
Un ERP (Planificacion de Recursos Empresariales) es el principal activo de software que tiene una empresa, ya que es un sistema de información que se encarga
de registrar las actividades de negocio, su enfoque debe estar orientado a la misionalidad de la organización. En un ERP básicamente se lleva  
la contabilidad, los presupuestos, la producción, ventas, recursos humanos, proyectos, etc. Ejemplos de ERP a usados a nivel local son SIIGO, DMS, STONE, SAP, Softland, etc.\\
Muchos errores directivos en las organizaciones es que no adquieren su software ERP para registrar y controlar sus actividades misionales, sino que se limitan al área 
administrativa (contabilidad,presupuestos, etc), generando desconexión entre los procesos y propiciando riesgos de desorden que hacen perder eficiencia a la organización.\\
La implementación de un software ERP dentro de una empresa es en general un proceso demorado y costoso, y su éxito se convierte en un factor crítico para que 
una compañía logre cumplir sus objetivos de negocio. 
\subsection{¿Que es OpenErp?}
OpenErp es un ERP de talla mundial, distribuido bajo la filosofia del software libre con la licencia AGPL.
La orientación de su desarrollo es módular, es decir que 
se instalan solamente las cosas que se necesitan concretamente para la organización, se puede comenzar instalando una aplicación y luego 
agregar otros módulos más adelante.\\
De esta manera los usuarios obtienen los beneficios de un software integrado. Los modulos desarrollados incluyen Ventas,
CRM (Customer Resource Management), gestión de proyectos, gestión de almacenes, fabricación, gestión financiera, recursos humanos sólo por nombrar algunos. 
Más de 700 módulos de OpenERP están disponibles en internet \cite{OpenErp}.

\subsection{¿Qué significa la licencia AGPL?}
La licencia  Pública General de GNU (GPL) es la vía legal que permite a los usuarios garantizar las cuatro libertades fundamentales del software libre.
La licencia pública general de Affero (en inglés, Affero General Public License, también Affero GPL o AGPL) es una licencia 
derivada de la Licencia (GPL) la cúal fue diseñada específicamente para asegurar la cooperación con la comunidad
en el caso de software que corra en servidores de red.\\
La Affero GPL es íntegramente una GNU GPL con una cláusula nueva que añade la obligación de distribuir el software si éste se 
ejecuta para ofrecer servicios a través de una red de ordenadores\cite{AGPLexp}.

\subsection {¿Qué es el Módulo OCS?}
El módulo Office Of Citizen Services, (Oficina de Atencion al Ciudadano) es un módulo orientado para organizaciones públicas que están obligadas a 
atender peticiones de la ciudadanía, el módulo sirve para registrar las PQR que 
ingresan por los diferentes canales.\\
Para consolidar el análisis de las PQR, se incluye un componente de georrefenciación, que permite que la información 
sea consultada desde un Sistema de Información Geográfica como Quantum o gvSIG.\\
No se incluye el componente de gestión documental. 