\documentclass[book]{upmethodology-document}

\setfrontcover{classic}
%\UseExtension{setlab}

\declaredocument{\LaTeX\ Packages for Unified Process Methodology}{Official Documentation}{UPM-2012-01}

\initialversion[1.0]{\makedate{11}{04}{2006}}{First release of this example}{\upmpublic}
\incversion{\makedate{19}{04}{2006}}{Add mtabular and mtable environments}{\upmpublic}
\incversion{\makedate{27}{04}{2006}}{Add package upmethodology-code}{\upmpublic}
\incversion{\makedate{13}{04}{2007}}{Update upmethodology-fmt package\newline Adding macros for informed people}{\upmpublic}
\incversion{\makedate{02}{07}{2007}}{Update upmethodology-fmt package\newline Adding symbols and text formatting functions\newline Updating footnote functions}{\upmpublic}
\incversion{\makedate{04}{07}{2007}}{Add front page selection macro}{\upmpublic}
\incversion{\makedate{07}{07}{2007}}{Add bibliography format macros}{\upmpublic}
\incversion{\makedate{08}{05}{2009}}{Fixing several mismatchs\newline Adding section of document extensions}{\upmpublic}
\incversion{\makedate{23}{06}{2009}}{Adding section on upmcaution, upminfo and upquestion environments}{\upmpublic}
\incversion{\makedate{09}{10}{2009}}{Add nodocumentinfo option in upmethodology-document class}{\upmpublic}
\incversion{\makedate{23}{10}{2009}}{Add back page package}{\upmpublic}
\incversion{\makedate{24}{10}{2009}}{Add extension package}{\upmpublic}
\incversion{\makedate{27}{10}{2009}}{Add extension macros and publication page}{\upmpublic}
\incversion{\makedate{28}{10}{2009}}{Fixing macro names in the \texttt{upmethodology-fmt} package. Add installation chapter.}{\upmpublic}
\incversion{\makedate{30}{10}{2009}}{Add document class explanations.}{\upmpublic}
\incversion{\makedate{02}{11}{2009}}{Add people name formatting macros. \newline Add package dependency list.}{\upmpublic}
\incversion{\makedate{03}{11}{2009}}{Add comment parameter for document role macros.}{\upmpublic}
\incversion{\makedate{21}{12}{2012}}{{\textbackslash}upmuse \ensuremath{\rightarrow} {\textbackslash}UseExtension}{\upmpublic}

\addauthorvalidator*[galland@arakhne.org]{St{\'e}phane}{Galland}{Original Author}

\setdockeywords{\LaTeX, Methodology Document Style, User Documentation, Extensible}

\setdocabstract{Packages upmethodology permis to write documents according to Unified Process Methodology or other methodology. It was initially wittren by St\'ephane \textsc{Galland} from Systems and Transport laboratory and is distributed by the \arakhneorg website.}

\makeatletter
\let\VERversion\upm@package@version@ver
\let\VERfmt\upm@package@fmt@ver
\let\VERdoc\upm@package@doc@ver
\let\VERfp\upm@package@fp@ver
\let\VERbp\upm@package@bp@ver
\let\VERext\upm@package@ext@ver
\let\VERtask\upm@package@task@ver
\let\VERdocclazz\upm@package@docclazz@ver
\let\VERcode\upm@package@code@ver
\makeatother

\setpublisher{Arakhn\^e.org Group}
\setcopyrighter{St\'ephane GALLAND}
\setprintingaddress{France}
\defupmlogo{arakhne_org_logo}
\defupmsmalllogo{small_arakhne_org_logo}

%\tracingmacros=2
%\tracingcommands=1

\begin{document}

\tableofcontents

\listoffigures

\listoftables

%###########################################################

\chapter{Introduction}

This set of package permits to write documents according to Unified Process Methodology. It was initially wittren by St\'ephane \textsc{Galland} from Systems and Transport laboratory\savefootnote{Systems and Transport Laboratory, University of Technology of Belfort-Montb{\'e}liard, France, \href{http://set.utbm.fr/}{http://set.utbm.fr/}}{foot:setlab} and is distributed by the \arakhneorg website.

Packages are:
\begin{itemize}
\item \texttt{upmethodology-version.sty}: permits to set the version and the status of the document. It also permits to manage the document history;
\item \texttt{upmethodology-fmt.sty}: provides some usefull functions to format the UP documents;
\item \texttt{upmethodology-document.sty}: provides functions to manage the project, the subproject and the status of the document;
\item \texttt{upmethodology-frontpage.sty}: formats and provides a front page for the document;
\item \texttt{upmethodology-backpage.sty}: formats and provides a back page for the document;
\item \texttt{upmethodology-task.sty}: is the \LaTeXe\xspace package that provides macros to manage project's tasks.
\item \texttt{upmethodology-document.cls}: is the \LaTeXe\xspace class that provides the whole document specification. It is based on \texttt{book} and on the previous packages;
\item \texttt{upmethodology-code.sty}: provides macros for source code formatting;
\item \texttt{upmethodology-extension.sty}: provides macros for extension mechanism.
\end{itemize}


%###########################################################
\part{General User Documentation}

\chapter{Download and Installation}

This chapter describes were to download \texttt{tex-upmethodology} and how to install it.

\section{Download}

\texttt{tex-upmethodology} is available on the \arakhneorg website: \url{http://www.arakhne.org/tex-upmethodology/}.

Different types of installation are available: manual installation, Debian packages.

\section{Manual System-wide Installation}

To make \texttt{tex-upmethodology} available to all users, copy the content of the \texttt{tex-upmethodology} archive inside one of your system texmf directory, usually one of:
\begin{itemize}
\item \texttt{/usr/share/texmf-texlive/tex/latex/upmethodology},
\item \texttt{/usr/share/texmf/tex/latex/upmethodology}.
\end{itemize}

The second is to rebuild the \LaTeX\ databases by invoking on a console (Unix syntax us used): \\
\texttt{\$> sudo mktexlsr}\\
\texttt{\$> sudo update-updmap --quiet}\\

\texttt{sudo} is a standard Linux tool which permits to authorized users to temporarely obtain the administration rights.

\section{Manual User-wide Installation}

To make \texttt{tex-upmethodology} available to one user, copy the content of the \texttt{tex-upmethodology} archive inside the \texttt{\$HOME/texmf} directory.

It is not required to rebuild the system-wide \LaTeX\ databases because the user's texmf are dynamically parsed by the \LaTeX\ distributions.

\section{Debian Package Installation}

Debian packages are available on \arakhneorg website: \url{http://www.arakhne.org/download.html}. Please follow the given rules.

\section{Package Dependencies}

This section contains the list of all the package dependencies for the \texttt{upmethodology} packages.

\subsection{upmethdology-backpage.sty}

\texttt{upmethodology-backpage} package depends on:
\begin{itemize}
\item \texttt{upmethodology-p-common}
\item \texttt{upmethodology-extension}
\end{itemize}

\subsection{upmethdology-code.sty}

\texttt{upmethodology-code} package depends on:
\begin{itemize}
\item \texttt{upmethodology-p-common}
\end{itemize}

\subsection{upmethdology-document.cls}

\texttt{upmethodology-document} class depends on:
\begin{itemize}
\item \texttt{upmethodology-p-common}
\item \texttt{a4wide}
\item \texttt{fancyhdr}
\item \texttt{upmethodology-document}
\item \texttt{upmethodology-extension}
\item \texttt{upmethodology-frontpage}
\item \texttt{upmethodology-backpage}
\item \texttt{upmethodology-task}
\item \texttt{upmethodology-code}
\item \texttt{upmethodology-spec}
\item \texttt{url}
\item \texttt{hyperref}
\end{itemize}

\subsection{upmethdology-document.sty}

\texttt{upmethodology-document} package depends on:
\begin{itemize}
\item \texttt{upmethodology-p-common}
\item \texttt{babel}
\item \texttt{vmargin}
\item \texttt{upmethodology-extension}
\item \texttt{upmethodology-fmt}
\item \texttt{upmethodology-version}
\end{itemize}

\subsection{upmethdology-extension.sty}

\texttt{upmethodology-extension} package depends on:
\begin{itemize}
\item \texttt{upmethodology-p-common}
\end{itemize}

\subsection{upmethdology-fmt.sty}

\texttt{upmethodology-fmt} package depends on:
\begin{itemize}
\item \texttt{upmethodology-p-common}
\item \texttt{graphicx}
\item \texttt{subfigure}
\item \texttt{tabularx}
\item \texttt{multicol}
\item \texttt{colortbl}
\item \texttt{picinpar}
\item \texttt{amsmath}
\item \texttt{pifont}
\item \texttt{setspace}
\item \texttt{txfonts}
\end{itemize}

\subsection{upmethdology-frontpage.sty}

\texttt{upmethodology-frontpage} package depends on:
\begin{itemize}
\item \texttt{upmethodology-p-common}
\item \texttt{upmethodology-extension}
\item \texttt{upmethodology-document}
\end{itemize}

\subsection{upmethdology-p-common.sty}

\texttt{upmethodology-p-common} package depends on:
\begin{itemize}
\item \texttt{ifthen}
\item \texttt{xspace}
\item \texttt{color}
\end{itemize}

\subsection{upmethdology-spec.sty}

\texttt{upmethodology-spec} package depends on:
\begin{itemize}
\item \texttt{upmethodology-p-common}
\item \texttt{ulem}
\item \texttt{upmethodology-fmt}
\item \texttt{upmethodology-code}
\end{itemize}

\subsection{upmethdology-task.sty}

\texttt{upmethodology-task} package depends on:
\begin{itemize}
\item \texttt{upmethodology-p-common}
\item \texttt{upmethodology-version}
\end{itemize}

\subsection{upmethdology-version.sty}

\texttt{upmethodology-version} package depends on:
\begin{itemize}
\item \texttt{upmethodology-p-common}
\item \texttt{upmethodology-fmt}
\end{itemize}

%###########################################################
\part{Package Documentation}

\chapter{Class upmethodology-document}

\begin{center}
	\texttt{Version: \VERdocclazz}
\end{center}

The \LaTeX\ class \texttt{upmethodology-document} provides the basic configuration for a document. According to an option, this class is able to extend the standard \texttt{book}, \texttt{report} or \texttt{article} \LaTeX\ classes. It also include a snipset of the other \texttt{upmethdology} packages.

\section{Types of documents}\label{section:documentclass:doctype}

\texttt{upmethodology-document} supports three special options which permits to set the type of document:
\begin{itemize}
\item \texttt{book}: A book-specification is a two-sided document composed of parts and chapters, and with a copyright page and document information page. This option indicates to \texttt{upmethodology-document} to load the \LaTeX\ standard \texttt{book} class. In addition the \texttt{{\textbackslash}part} and \texttt{{\textbackslash}chapter} macros are supported, and the following macros are automatically expanded: \texttt{{\textbackslash}makefrontcover}, \texttt{{\textbackslash}upmpublicationpage}, \texttt{{\textbackslash}upmdocumentsummary}, \texttt{{\textbackslash}makebackcover}. This behaviour may be overridden by the other class options.

\item \texttt{report}: A report-specification is a one-sided document composed of chapters (no part), and with a document information page. This option indicates to \texttt{upmethodology-document} to load the \LaTeX\ standard \texttt{report} class. In addition the \texttt{{\textbackslash}part} macro is ignored\savefootnote{The macro is redefined to print a warning message when used, no error message is generated.}{ignoretexmacro} and \texttt{{\textbackslash}chapter} macro is supported, and the following macros are automatically expanded: \texttt{{\textbackslash}makefrontcover}, \texttt{{\textbackslash}upmdocumentsummary}, \texttt{{\textbackslash}makebackcover}. This behaviour may be overridden by the other class options.

\item \texttt{article}: An article-specification is a one-sided document composed of sections (no part nor chapter). This option indicates to \texttt{upmethodology-document} to load the \LaTeX\ standard \texttt{article} class. In addition the \texttt{{\textbackslash}part}and \texttt{{\textbackslash}chapter} macros are ignored\reffootnote{ignoretexmacro}, and the following macros are automatically expanded: \texttt{{\textbackslash}makefrontcover}, \texttt{{\textbackslash}makebackcover}. This behaviour may be overridden by the other class options.

\end{itemize}

\section{Class options}

Table~\tabref{documentclassoptions} contains the options supported by \texttt{upmethodology-document}. Any option not explicitely supported by the class is directly pass to the underlying standard \LaTeX\ class (\texttt{book}, \texttt{report} or \texttt{article} according to the type of document, see~\ref{section:documentclass:doctype}).

\begin{mtable}{\linewidth}{2}{|l|X|}{Options of \texttt{upmethodology-document} class}{documentclassoptions}
\captionastitle
\tabularheader{Option}{Explanation}

\hline

book & see section~\ref{section:documentclass:doctype}. \\
\hline
report & see section~\ref{section:documentclass:doctype}. \\
\hline
article & see section~\ref{section:documentclass:doctype}. \\

\hline\hline

oneside & the document is generated assuming that each page will be printed on its recto side. This option overrides any previous occurrence of \texttt{twoside} option. \\
\hline
twoside & the document is generated assuming that each page will be printed on both recto and verso sides. This option overrides any previous occurrence of \texttt{oneside} option. \\

\hline\hline

francais & same as \texttt{french}. \\
\hline
french & the document is written in French. \texttt{upmethodology} packages use the French translations for the generated texts.  This option overrides any previous occurrence of \texttt{english} option. \\
\hline
english & the document is written in English. \texttt{upmethodology} packages use the English translations for the generated texts.  This option overrides any previous occurrence of \texttt{french} option. \\

\hline\hline

documentinfo & invoke \texttt{{\textbackslash}upmdocumentsummary}, \texttt{{\textbackslash}upmdocumentauthors}, \texttt{{\textbackslash}upmdocumentvalidators}, \texttt{{\textbackslash}upmdocumentinformedpeople}, and \texttt{{\textbackslash}upmhistory} macros at the begining of the document. This option overrides any previous occurrence of \texttt{nodocumentinfo} option. \\
\hline
nodocumentinfo & do not invoke \texttt{{\textbackslash}upmdocumentsummary}, \texttt{{\textbackslash}upmdocumentauthors}, \texttt{{\textbackslash}upmdocumentvalidators}, \texttt{{\textbackslash}upmdocumentinformedpeople}, nor \texttt{{\textbackslash}upmhistory} macros at the begining of the document. This option overrides any previous occurrence of \texttt{documentinfo} option. \\

\hline\hline

pubpage & invoke \texttt{{\textbackslash}upmpublicationpage} macro at the begining of the document. This option overrides any previous occurrence of \texttt{nopubpage} option. \\
\hline
nopubpage & do not invoke \texttt{{\textbackslash}upmpublicationpage} macro at the begining of the document. This option overrides any 
previous occurrence of \texttt{pubpage} option. \\

\hline
\end{mtable}

\section{Additional Features}

\texttt{upmethodology-document} provides a constant behaviour for all types of document:
\begin{itemize}
\item \texttt{hyperref} is loaded and set with the document informations;
\item \texttt{{\textbackslash}setpdfcolor} is redefined and linked to \texttt{hyperref};
\item \texttt{fancyhdr} is loaded and default page header and page footer is provided.
\end{itemize}

%###########################################################

\chapter{Package upmethodology-version}

\begin{center}
	\texttt{Version: \VERversion}
\end{center}

The package \texttt{upmethodology-version} permits to set the version and the status of the document. It also provides functions to manage the document history;

\section{Constants for the Document Status}

Some \LaTeXe\xspace variables provides strings that describe the status of the document. They can be used in functions such as \texttt{{\textbackslash}updateversion}.
\begin{itemize}
\item \texttt{{\textbackslash}upmrestricted}: the document is under a restricted access, generally corresponding to the list of authors;
\item \texttt{{\textbackslash}upmvalidable}: authors indicates with this tathat the document could be sent to validators;
\item \texttt{{\textbackslash}upmvalidated}: the document was validated, but not published and published by all people;
\item \texttt{{\textbackslash}upmpublic}: the document published and accessible by all people;
\end{itemize}

\subsection{Information about the Document}

The following functions permit to access to the informations about the document:
\begin{itemize}
\item \texttt{{\textbackslash}theupmversion}: replies the last version number for the document;

\item \texttt{{\textbackslash}upmdate\{version\}}: replies the updating date of the document corresponding to the given version number;

\item \texttt{{\textbackslash}upmdescription\{version\}}: replies the updating comment of the document corresponding to the given version number;

\item \texttt{{\textbackslash}upmstatus\{version\}}: replies the status of the document corresponding to the given version number.

\item \texttt{{\textbackslash}theupmdate}: replies the last updating date for the document. It is equivalent to \texttt{{\textbackslash}upmdate\{{\textbackslash}theupmversion\}};

\item \texttt{{\textbackslash}theupmlastmodif}: replies the last updating comment for the document. It is equivalent to \texttt{{\textbackslash}upmdescription\{{\textbackslash}theupmversion\}};

\item \texttt{{\textbackslash}theupmstatus}: replies the last status for the document. It is equivalent to \texttt{{\textbackslash}upmstatus\{{\textbackslash}theupmversion\}};
\end{itemize}

\section{Register Revisions}

The package \texttt{upmethodology-version} permits to register revisions for building an history. The available functions are:
\begin{itemize}
\item \texttt{{\textbackslash}updateversion\{version\}\{date\}\{description\}\{status\}}: registers an revision for the document. The revision indicates that the given version was produced at the given date. A small description of the changes and the resulting document's status must be also provided. The function \texttt{{\textbackslash}updateversion} is a generalization of the following functions;

\item \texttt{{\textbackslash}initialversion[version]\{date\}\{description\}\{status\}}: registers the initial version of the document. If not given, the version is assumed to be \texttt{0.1};

\item \texttt{{\textbackslash}incversion\{date\}\{description\}\{status\}}: regiters a revision corresponding to the next major version. For example, if the version number was \texttt{2.67} before \texttt{{\textbackslash}incversion}, this function add the version \texttt{3.67} with the given informations (incrementation of the major part of the version number);

\item \texttt{{\textbackslash}incsubversion\{date\}\{description\}\{status\}}: regiters a revision corresponding to the next minor version. For example, if the version number was \texttt{2.67} before \texttt{{\textbackslash}incsubversion}, this function add the version \texttt{2.68} with the given informations (incrementation of the minor part of the version number);
\end{itemize}

\section{Formatted List of Versions}

To obtain a formatted list of versions, you could use the macro \texttt{{\textbackslash}upmhistory[width]} which produces:

\upmhistory

\section{Localization}

The following macros defines some localized strings used by \texttt{upmethodology-version}:
\begin{itemize}
\item \texttt{{\textbackslash}upm@lang@date}: Date;
\item \texttt{{\textbackslash}upm@lang@updates}: Updates;
\item \texttt{{\textbackslash}upm@lang@version}: Version;
\item \texttt{{\textbackslash}upm@lang@version@history}: Version History;
\end{itemize}

%###########################################################

\chapter{Package upmethodology-fmt}

\begin{center}
	\texttt{Version: \VERfmt}
\end{center}

The package \texttt{upmethodology-fmt} provides some usefull facilities to format an UP document.

\section{Figures}

You could include afigure inside your document with the following macros: \\
\texttt{{\textbackslash}mfigure[position]\{options\}\{filename\}\{caption\}\{label\}} \\
\texttt{{\textbackslash}mfigure*[position]\{options\}\{filename\}\{caption\}\{label\}} \\

These two macros permits to include an image in your document.The parameters are:
\begin{itemize}
\item \texttt{position}: is the desired position of the figure (see {\textbackslash}begin{figure}[position]). It could be \texttt{t} (top of the page), \texttt{b} (bottom of the page), \texttt{h} (at the macro location if possible) or \texttt{H} (at macro location);

\item \texttt{options}: are the options passed to \texttt{{\textbackslash}includegraphics};

\item \texttt{filename}: is the filename passed to \texttt{{\textbackslash}includegraphics};

\item \texttt{caption}: is the caption of the figure (see {\textbackslash}caption\{caption\});

\item \texttt{label}: is the label used to reference the figure (see {\textbackslash}label\{fig:label\}).
\end{itemize}

The difference between \texttt{{\textbackslash}mfigure} and \texttt{{\textbackslash}mfigure*} is the same as the difference between \texttt{{\textbackslash}begin\{figure\}} and \texttt{{\textbackslash}begin\{figure*\}}: the star-version fits to the entire paper width event if the document has two or more columns.

Because the two macros abose register a label with string starting with \texttt{fig:}, we propose the following function to easily access to the figure's references:
\begin{itemize}
\item \texttt{{\textbackslash}figref\{label\}}: is equivalent to \texttt{{\textbackslash}ref\{fig:label\}};
\item \texttt{{\textbackslash}figpageref\{label\}}: is equivalent to \texttt{{\textbackslash}pageref\{fig:label\}}.
\end{itemize}

The figure~\figref{example:mfigure} page~\figpageref{example:mfigure} is obtained with the macro: \texttt{{\textbackslash}mfigure[ht]\{width=.6{\textbackslash}linewidth\}\{slogo\}\{Example of figure inclusion with {\textbackslash}texttt\{\{{\textbackslash}textbackslash\}mfigure\}\}\{example:mfigure\}}. The reference and page reference are obtained with \texttt{{\textbackslash}figref\{example:mfigure\}} and \texttt{{\textbackslash}figpageref\{example:mfigure\}}.

\mfigure[ht]{width=.6\linewidth}{arakhne_org_logo}{Example of figure inclusion with \texttt{{\textbackslash}mfigure}}{example:mfigure}

\section{Sub-figures}

In some case, it is usefull to put several images inside the same picture, but without lousing the possibility to reference each subfigure. This feature was proposed by the package \texttt{subfigure}. The following environments provides helper functions for \texttt{subfigure}:
\texttt{{\textbackslash}begin\{mfigures\}[position]\{caption\}\{label\}\\
...\\
{\textbackslash}end\{mfigures\}} \\
\texttt{{\textbackslash}begin\{mfigures*\}[position]\{caption\}\{label\}\\
...\\
{\textbackslash}end\{mfigures*\}} \\

These two macros permits to include an image in your document.The parameters are:
\begin{itemize}
\item \texttt{position}: is the desired position of the figure (see {\textbackslash}begin{figure}[position]). It could be \texttt{t} (top of the page), \texttt{b} (bottom of the page), \texttt{h} (at the macro location if possible) or \texttt{H} (at macro location);

\item \texttt{caption}: is the caption of the figure (see {\textbackslash}caption\{caption\});

\item \texttt{label}: is the label used to reference the figure (see {\textbackslash}label\{fig:label\}).
\end{itemize}

Inside the environment \texttt{{\textbackslash}mfigures[*]}, you could use the macro \texttt{{\textbackslash}mfigure} to properly include a subfigure (the first optional parameter is ignored) or you could use the macro \texttt{{\textbackslash}msubfigure\{options\}\{file\}\{caption\}}.

The figure~\figref{example:msubfigure} page~\figpageref{example:msubfigure} is obtained with the environment:\\
\texttt{{\textbackslash}begin\{mfigures\}\{Example of subfigures with {\textbackslash}texttt\{mfigures\}\}\{example:msubfigure\}}\\
\texttt{{\textbackslash}mfigure\{width=.4{\textbackslash}linewidth\}\{logo\}\{First subfigure\}\{example:firstsubfigure\}} \\
\texttt{{\textbackslash}hfill} \\
\texttt{{\textbackslash}msubfigure\{width=.4{\textbackslash}linewidth\}\{smalllogo\}\{Second subfigure\}} \\
\texttt{{\textbackslash}end\{mfigures\}}

The reference and page reference are obtained with \texttt{{\textbackslash}figref\{example:msubfigure\}} and \texttt{{\textbackslash}figpageref\{example:msubfigure\}}.

\begin{mfigures}{Example of subfigures with \texttt{mfigures}}{example:msubfigure}
	\mfigure{width=.4\linewidth}{arakhne_org_logo}{First subfigure}{example:firstsubfigure}
	\hfill
	\msubfigure{width=.4\linewidth}{small_arakhne_org_logo}{Second subfigure}
\end{mfigures}

The references to the subfigures could be obtained in two way:\nopagebreak\begin{itemize}
\item using the label given as the last parameter of \texttt{{\textbackslash}mfigure}, eg. the label \texttt{example:firstsubfigure} corresponds to \figref{example:firstsubfigure};
\item using the label of the enclosing figure to which the index of the subfigure could be appended (in its roman representation and prefixed by the character ``\texttt{:}''), eg. the label \texttt{example:msubfigure:b} corresponds to \figref{example:msubfigure:b};
\end{itemize}

\section{Figures with embedded \TeX\xspace macros}

In several case it is usefull to include \TeX\xspace macros inside a figure. It is possible with the files \texttt{.pstex} exporting from a software such as \texttt{xfig}.

To put a \TeX\xspace macro inside your figure, follows the steps:
\begin{enumerate}
\item in \texttt{xfig} create a text label with the \emph{special} property set. In this label types the string \texttt{{\textbackslash}FIG$\delta$} where $\delta$ must be replaced by an identifier of your choice but only composed of letters (example: \texttt{{\textbackslash}FIGmyid});
\item in \texttt{xfig} saves your figure as \texttt{.pstex} files;
\item in \LaTeX, just before including the figure with the embedded \TeX\xspace macros, define the expressions to put in the figure. This action must be done with one of the macros:
	\begin{itemize}
	\item \texttt{{\textbackslash}figmath\{id\}\{expr\}} will associate to the given identifier the given mathematical expression,
	\item \texttt{{\textbackslash}figtext\{id\}\{expr\}} will associate to the given identifier the given text expression;
	\end{itemize}
\item in \LaTeX, include the figure with one of the macros: \\
\texttt{{\textbackslash}mfigurewtex[position]\{width\}\{filename\}\{caption\}\{label\}} \\
\texttt{{\textbackslash}mfigurewtex*[position]\{width\}\{filename\}\{caption\}\{label\}}
\end{enumerate}

\section{Tabulars}

You could include a tabular inside your document with the following environment: \\
\texttt{{\textbackslash}begin\{mtabular\}[width]\{ncolumns\}\{columns\}...{\textbackslash}end\{mtabular\}} \\

This tabular is an extension of the \texttt{tabularx} environment which provides dynamic columns with the specifier \texttt{X}. The parameters are:
\begin{itemize}
\item \texttt{width}: is the desired width of the tabular;

\item \texttt{ncolumns}: is the count of columns in the tabular. It must be consistent with the column description;

\item \texttt{columns}: is the description of the columns according to the \texttt{tabular} and \texttt{tabularx} packages.

\end{itemize}

The \texttt{mtabular} environment provides:
\begin{itemize}
\item \texttt{{\textbackslash}tabulartitle\{title\}} \\
	this macro permits to define the title of the tabular. It uses the colors \texttt{backtableheader} and \texttt{fronttableheader} for the background and the foreground respectively;

\item \texttt{{\textbackslash}tabularheader\{$header_1$\}...\{$header_n$\}} \\
	this macro permits to define the titles of the columns. It uses the colors \texttt{backtableheader} and \texttt{fronttableheader} for the background and the foreground respectively. Because the count of columns was given to the environment this function takes the same count of parameters as the count of columns.
\end{itemize}

The following example of table is obtained by: \\
\texttt{{\textbackslash}begin\{mtabular\}[{\textbackslash}linewidth]\{4\}\{lXrX\}} \\
\texttt{{\textbackslash}tabulartitle\{Example of {\textbackslash}texttt\{mtabular\}\}} \\
\texttt{{\textbackslash}tabularheader\{Col1\}\{Col2\}\{Col3\}\{Col4\}} \\
\texttt{a \& b \& c \& d {\textbackslash}{\textbackslash}} \\
\texttt{{\textbackslash}hline} \\
\texttt{e \& f \& g \& h {\textbackslash}{\textbackslash}} \\
\texttt{{\textbackslash}end\{mtabular\}}

\begin{mtabular}[\linewidth]{4}{lXrX}
	\tabulartitle{Example of \texttt{mtabular}}
	\tabularheader{Col1}{Col2}{Col3}{Col4}
	a & b & c & d \\
	\hline
	e & f & g & h \\
\end{mtabular}

\section{Tables}

You could include a table inside your document with the following environment: \\
\texttt{{\textbackslash}begin\{mtable\}[position]\{width\}\{ncolumns\}\{columns\}\{caption\}\{label\}...{\textbackslash}end\{mtable\}} \\

This environment is based on the \texttt{mtabular} environment. The parameters are:
\begin{itemize}
\item \texttt{position}: is the desired position of the table according to the \LaTeX's \texttt{table} definition;
\item \texttt{width}: is the desired width of the table (ie., the tabular inside the table);
\item \texttt{ncolumns}: is the count of columns in the table (ie., the tabular inside the table). It must be consistent with the column description;
\item \texttt{columns}: is the description of the columns according to the \texttt{tabular} and \texttt{tabularx} packages;
\item \texttt{caption}: is the caption of the table;
\item \texttt{label}: is the label referencing the table.
\end{itemize}

Because the \texttt{mtable} environment registers a label with a string starting with \texttt{tab:}, the following functions are proposed to easily access to the table's references:
\begin{itemize}
\item \texttt{{\textbackslash}tabref\{label\}}: is equivalent to \texttt{{\textbackslash}ref\{tab:label\}};
\item \texttt{{\textbackslash}tabpageref\{label\}}: is equivalent to \texttt{{\textbackslash}pageref\{tab:label\}}.
\end{itemize}

The table~\tabref{example:mtable} page~\tabpageref{example:mtable} is an illustration of the following \LaTeX\ code: \\
\texttt{{\textbackslash}begin\{mtable\}\{{\textbackslash}linewidth\}\{4\}\{lXrX\}\{Example of {\textbackslash}texttt{mtable}\}\{example:mtable\}} \\
%\texttt{{\textbackslash}tabulartitle\{Example of {\textbackslash}texttt\{mtable\}\}} \\
\texttt{{\textbackslash}captionastitle} \\
\texttt{{\textbackslash}tabularheader\{Col1\}\{Col2\}\{Col3\}\{Col4\}} \\
\texttt{a \& b \& c \& d {\textbackslash}{\textbackslash}} \\
\texttt{{\textbackslash}hline} \\
\texttt{e \& f \& g \& h {\textbackslash}{\textbackslash}} \\
\texttt{{\textbackslash}end\{mtable\}}

\begin{mtable}{\linewidth}{4}{lXrX}{Example of \texttt{mtable}}{example:mtable}
	%\tabulartitle{Example of \texttt{mtable}}
	\captionastitle
	\tabularheader{Col1}{Col2}{Col3}{Col4}
	a & b & c & d \\
	\hline
	e & f & g & h \\
\end{mtable}

The macro \texttt{{\textbackslash}captionastitle} is equivalent to a call to the macro \texttt{{\textbackslash}tabulartitle} with the caption in parameter.


\section{Enumerations}

The package \texttt{upmethodology-fmt} provides a set of macros dedicated to enumeration lists.

\subsection{Enumeration Counters}

Sometimes it is usefull to start an enumeration list from a specifical given number. This package provides several macros for saving and restoring the counter use by the enumeration lists.

Caution: only once counter could be saved at a given time.

\begin{itemize}
\item \texttt{{\textbackslash}savecounter\{name\}} \\
	save the counter identifier by the given name;
\item \texttt{{\textbackslash}restorecounter\{name\}} \\
	put the previously saved value into the given counter;
\item \texttt{{\textbackslash}setenumcounter\{value\}} \\
	force the value of the enumeration counter;
\item \texttt{{\textbackslash}getenumcounter} \\
	replies the value of the enumeration counter;
\item \texttt{{\textbackslash}saveenumcounter} \\
	save the enumeration counter;
\item \texttt{{\textbackslash}restoreenumcounter} \\
	force the enumeration to use the saved counter's value;
\end{itemize}

\subsection{Inline Enumeration}

In several document, an enumeration of things is written inside a paragraph instead of inside a list of points. A simple example is:
\begin{inlineenumeration}
\item first thing;
\item second thing;
\item etc.
\end{inlineenumeration}
And it is produced by the \LaTeX~code: \\
\texttt{{\textbackslash}begin\{inlineenumeration\}}\\
\texttt{{\textbackslash}item first thing;}\\
\texttt{{\textbackslash}item second thing;}\\
\texttt{{\textbackslash}item etc.}\\
\texttt{{\textbackslash}end\{inlineenumeration\}}\\

\section{Footnotes}

The package \texttt{upmethodology-fmt} provides a set of macros allowing to save the reference number of a footnote and to recall this reference many time as required.

\begin{itemize}
\item \texttt{{\textbackslash}savefootnote\{footnote text\}\{footnote id\}} \\
	put a footnote and mark it with the corresponding label. \\
	Example: \texttt{{\textbackslash}savefootnote\{This is an example of a recallable footnote\}\{footrecalla\}}\savefootnote{This is an example of a recallable footnote}{footrecalla};
\item \texttt{{\textbackslash}savefootnote*\{footnote text\}\{footnote id\}} \\
	mark a footnote with the corresponding label but do not put in the current page. \\
	Example 1: \texttt{{\textbackslash}savefootnote*\{This is a second example of a recallable footnote\}\{footrecallb\}}\savefootnote*{This is a second example of a recallable footnote}{footrecallb}; \\
	Example 2: \texttt{{\textbackslash}savefootnote*\{This is a third example of a recallable footnote\}\{footrecallc\}}\savefootnote*{This is a third example of a recallable footnote}{footrecallc}.
\item \texttt{{\textbackslash}reffootnote\{footnote id\}} \\
	recall the footnote reference without page number. \\
	Example 1: \texttt{{\textbackslash}reffootnote\{footrecalla\}}\reffootnote{footrecalla}$=B$; \\
	example 2: \texttt{{\textbackslash}reffootnote\{footrecallb\}}\reffootnote{footrecallb}$=A$; \\
	example 4: \texttt{{\textbackslash}reffootnote\{footrecalld\}}\reffootnote{footrecalld}$=?$.
\item \texttt{{\textbackslash}reffootnote*\{footnote id\}} \\
	recall the footnote reference with the page number if different of the current page. \\
	Example 1: \texttt{{\textbackslash}reffootnote*\{footrecalla\}}\reffootnote*{footrecalla}; \\
	example 2: \texttt{{\textbackslash}reffootnote*\{footrecallb\}}\reffootnote*{footrecallb}; \\
	example 3: \texttt{{\textbackslash}reffootnote*\{footrecallc\}}\reffootnote*{footrecallc}; \\
	example 4: \texttt{{\textbackslash}reffootnote*\{footrecalld\}}\reffootnote*{footrecalld}.
\end{itemize}

\section{UML diagrams on the side of paragraphs}

The package \texttt{upmethodology-fmt} provides an environment which permits to put an UML diagram (or any other picture) on the side of a paragraph.

\begin{itemize}
\item \texttt{{\textbackslash}begin\{umlinpar\}[width]\{picture\_path\}}\\\texttt{text}\\\texttt{{\textbackslash}end\{umlinpat\}} \\
	put the specified picture on the side of the given text. The optional parameter \texttt{width} corresponds to the desired width ofthe picture. By default it is \texttt{.5{\textbackslash}linewidth}.
\end{itemize}

\begin{umlinpar}{small_arakhne_org_logo}
	This paragraph is an typical example of the usage of the environment \texttt{umlinpar}. To obtain it, the following \LaTeX\ code was typed: \\
\texttt{{\textbackslash}begin\{umlinpar\}\{smalllogo\}} \\
\texttt{This paragraph is an typical example of the usage of the environment {\textbackslash}texttt\{umlinpar\}.} \\
\texttt{{\textbackslash}end\{umlinpar\}} \\
\end{umlinpar}

\section{Date formatting}

Because the concept of date was important and unfortunately localized, this package provides a set of functions to define and extract information from dates (the supported date formats are described in table~\ref{tab:date.formats}):

\begin{itemize}
\item \texttt{{\textbackslash}makedate\{day\}\{month\}\{year\}}\\
permits to create the text corresponding to the given date according to the current localized date format.
\item \texttt{{\textbackslash}extractyear\{formatted\_date\}}\\
extract the year field from a date respecting the localized date format.
\item \texttt{{\textbackslash}extractmonth\{formatted\_date\}}\\
extract the month field from a date respecting the localized date format.
\item \texttt{{\textbackslash}extractday\{formatted\_date\}}\\
extract the day field from a date respecting the localized date format.
\end{itemize}

\begin{table}[Hht]
	\begin{center}
		\begin{tabular}{|>{\ttfamily}l|l|}
		\hline
		yyyy/mm/dd & default format \\
		%mm/dd/yyyy & US format \\
		dd/mm/yyyy & french format \\
		\hline
		\end{tabular}
	\end{center}
	\caption{List of supported date formats}
	\label{tab:date.formats}
\end{table}

\section{Text formatting}

The package \texttt{upmethodology-fmt} provides a set of macros to format the text.

\begin{itemize}
\item \texttt{{\textbackslash}textsup\{text\}} \\
	put a text as exponent in text mode instead of the basic \LaTeX exponent in math mode. \\
	Example: \texttt{{\textbackslash}textsup\{this is an exponent\}}\textsup{this is an exponent};
\item \texttt{{\textbackslash}textsub\{text\}} \\
	put a text as indice in text mode instead of the basic \LaTeX indice in math mode. \\
	Example: \texttt{{\textbackslash}textsub\{this is an indice\}}\textsub{this is an indice};
\item \texttt{{\textbackslash}makename[von]\{first name\}\{last name\}} \\
	format the specified people name components according to the document standards. By default, the format \texttt{first von last} is used. \\
	Example: \texttt{{\textbackslash}makename[von]\{Ludwig Otto Frederik Wilhelm\}\{Wittelsbach\}},\\
		``\makename[von]{Ludwig Otto Frederik Wilhelm}{Wittelsbach}'';
\item \texttt{{\textbackslash}upmmakename[von]\{first name\}\{last name\}\{separator\}} \\
	format the specified people name components according to the document standards. By default, the format \texttt{first von last} is used. \\
	Example: \texttt{{\textbackslash}upmmakename[von]\{Ludwig Otto Frederik Wilhelm\}\{Wittelsbach\}\{/\}}, \\
		``\upmmakename[von]{Ludwig Otto Frederik Wilhelm}{Wittelsbach}{/}'';
\item \texttt{{\textbackslash}prname[von]\{first name\}\{last name\}} \\
      \texttt{{\textbackslash}prname*[von]\{first name\}\{last name\}} \\
	format the specified people name components according to the document standards for \emph{Professor} title. By default, the format \texttt{first von last} is used. The star-ed version is post-fixed, the non-star-ed version is prefixed. \\
	Example 1: \texttt{{\textbackslash}prname\{Pierre\}\{Martin\}}, ``\prname{Pierre}{Martin}''; \\
	Example 2: \texttt{{\textbackslash}prname*\{Pierre\}\{Martin\}}, ``\prname*{Pierre}{Martin}'';
\item \texttt{{\textbackslash}drname[von]\{first name\}\{last name\}} \\
      \texttt{{\textbackslash}drname*[von]\{first name\}\{last name\}} \\
	format the specified people name components according to the document standards for \emph{Doctor} title. By default, the format \texttt{first von last} is used. The star-ed version is post-fixed, the non-star-ed version is prefixed. \\
	Example 1: \texttt{{\textbackslash}drname\{Pierre\}\{Martin\}}, ``\drname{Pierre}{Martin}''; \\
	Example 2: \texttt{{\textbackslash}drname*\{Pierre\}\{Martin\}}, ``\drname*{Pierre}{Martin}'';
\item \texttt{{\textbackslash}phdname[von]\{first name\}\{last name\}} \\
      \texttt{{\textbackslash}phdname*[von]\{first name\}\{last name\}} \\
	format the specified people name components according to the document standards for \emph{Philosophi\ae Doctor} title. By default, the format \texttt{first von last} is used. The star-ed version is post-fixed, the non-star-ed version is prefixed. \\
	Example 1: \texttt{{\textbackslash}phdname\{Pierre\}\{Martin\}}, ``\phdname{Pierre}{Martin}''; \\
	Example 2: \texttt{{\textbackslash}phdname*\{Pierre\}\{Martin\}}, ``\phdname*{Pierre}{Martin}'';
\item \texttt{{\textbackslash}scdname[von]\{first name\}\{last name\}} \\
      \texttt{{\textbackslash}scdname*[von]\{first name\}\{last name\}} \\
	format the specified people name components according to the document standards for \emph{Scienti\ae Doctor} title. By default, the format \texttt{first von last} is used. The star-ed version is post-fixed, the non-star-ed version is prefixed. \\
	Example 1: \texttt{{\textbackslash}scdname\{Pierre\}\{Martin\}}, ``\scdname{Pierre}{Martin}''; \\
	Example 2: \texttt{{\textbackslash}scdname*\{Pierre\}\{Martin\}}, ``\scdname*{Pierre}{Martin}'';
\item \texttt{{\textbackslash}mdname[von]\{first name\}\{last name\}} \\
      \texttt{{\textbackslash}mdname*[von]\{first name\}\{last name\}} \\
	format the specified people name components according to the document standards for \emph{Medicin\ae Doctor} title. By default, the format \texttt{first von last} is used. The star-ed version is post-fixed, the non-star-ed version is prefixed. \\
	Example 1: \texttt{{\textbackslash}mdname\{Pierre\}\{Martin\}}, ``\mdname{Pierre}{Martin}''; \\
	Example 2: \texttt{{\textbackslash}mdname*\{Pierre\}\{Martin\}}, ``\mdname*{Pierre}{Martin}'';
\item \texttt{{\textbackslash}pengname[von]\{first name\}\{last name\}} \\
      \texttt{{\textbackslash}pengname*[von]\{first name\}\{last name\}} \\
	format the specified people name components according to the document standards for \emph{Professional/Chartered Engineer} title. By default, the format \texttt{first von last} is used. The star-ed version is post-fixed, the non-star-ed version is prefixed. \\
	Example 1: \texttt{{\textbackslash}pengname\{Pierre\}\{Martin\}}, ``\pengname{Pierre}{Martin}''; \\
	Example 2: \texttt{{\textbackslash}pengname*\{Pierre\}\{Martin\}}, ``\pengname*{Pierre}{Martin}'';
\item \texttt{{\textbackslash}iengname[von]\{first name\}\{last name\}} \\
      \texttt{{\textbackslash}iengname*[von]\{first name\}\{last name\}} \\
	format the specified people name components according to the document standards for \emph{Incorporated Engineer} title. By default, the format \texttt{first von last} is used. The star-ed version is post-fixed, the non-star-ed version is prefixed. \\
	Example 1: \texttt{{\textbackslash}iengname\{Pierre\}\{Martin\}}, ``\iengname{Pierre}{Martin}''; \\
	Example 2: \texttt{{\textbackslash}iengname*\{Pierre\}\{Martin\}}, ``\iengname*{Pierre}{Martin}''.
\end{itemize}
\section{Symbols}

The package \texttt{upmethodology-fmt} provides several symbols described inside the table~\ref{tab:symbols}.

\begin{table}[Hht]
	\begin{center}
		\begin{tabular}{|>{\ttfamily}l|l|}
		\hline
		{\textbackslash}arakhneorg & \arakhneorg \\
		{\textbackslash}copyright & \copyright \\
		{\textbackslash}trademark & \trademark \\
		{\textbackslash}regmark & \regmark \\
		{\textbackslash}smalltrade & \smalltrade \\
		{\textbackslash}smallreg & \smallreg \\
		{\textbackslash}smallcopy & \smallcopy \\
		{\textbackslash}ust & \ust \\
		{\textbackslash}und & \und \\
		{\textbackslash}urd & \urd \\
		{\textbackslash}uth & \uth \\
		\hline
		\end{tabular}
	\end{center}
	\caption{List of symbols}
	\label{tab:symbols}
\end{table}

\section{Bibliography}

The package \texttt{upmethodology-fmt} provides a set of macros allowing to manage the bibliography. The default bibliography style is \texttt{abbr}.

\begin{itemize}
\item \texttt{{\textbackslash}bibliographystyle\{style\}} \\
	set the bibliography style to use. \\
	Example: \texttt{{\textbackslash}bibliographystyle\{alpha\}};
\item \texttt{{\textbackslash}bibliography\{file\}} \\
	set the \textsc{Bib}\TeX\ file to use. \\
	Example: \texttt{{\textbackslash}bibliography\{mybib\}};
\item \texttt{{\textbackslash}bibsize\{size\}} \\
	set the font size used for the bibliography section. \\
	Example: \texttt{{\textbackslash}bibsize\{{\textbackslash}Huge\}};
\end{itemize}

\section{Framed Mini Pages}

Standard \LaTeX\ distribution provides the \texttt{minipage} environment. This environment permits to put a small piece of page inside your document. Package \texttt{upmethodology-fmt} provides a framed extension of the original \texttt{minipage} environment:
\par
\begin{tabularx}{\linewidth}{XX}
\texttt{{\textbackslash}begin\{framedminipage\}{[}{\textbackslash}linewidth{]}} \newline
\texttt{This is an example of a framed minipage.} \newline
\texttt{{\textbackslash}end\{framedminipage\}}
&
\begin{framedminipage}{\linewidth}This is an example of a framed minipage.\end{framedminipage}
\\
\end{tabularx}

\section{Message Boxes}

The package \texttt{upmethodology-fmt} provides a set of environment to put emphasis message boxes in the text. Three types of boxes are supported: caution, information, and question.

\par
\begin{tabularx}{\linewidth}{XX}

\texttt{{\textbackslash}begin\{upmcaution\}{[}width{]}} \newline
\texttt{This is an example of a caution message. This text must be rendered with enough height (usually 2 lines of text) to avoid intersection between the caution icon and the box frame.} \newline
\texttt{{\textbackslash}end\{upmcaution\}}
&
\begin{upmcaution}This is an example of a caution message. This text must be rendered with enough height (usually 2 lines of text) to avoid intersection between the caution icon and the box frame.\end{upmcaution}
\\
	
\texttt{{\textbackslash}begin\{upminfo\}{[}width{]}} \newline
\texttt{This is an example of an information message. This text must be rendered with enough height (usually 2 lines of text) to avoid intersection between the caution icon and the box frame.} \newline
\texttt{{\textbackslash}end\{upminfo\}}
&
\begin{upminfo}This is an example of an information message. This text must be rendered with enough height (usually 2 lines of text) to avoid intersection between the caution icon and the box frame.\end{upminfo}
\\

\texttt{{\textbackslash}begin\{upmquestion\}{[}width{]}} \newline
\texttt{This is an example of a question message. This text must be rendered with enough height (usually 2 lines of text) to avoid intersection between the caution icon and the box frame.} \newline
\texttt{{\textbackslash}end\{upmquestion\}}
&
\begin{upmquestion}This is an example of a question message. This text must be rendered with enough height (usually 2 lines of text) to avoid intersection between the caution icon and the box frame.\end{upmquestion}
\\

\end{tabularx}

\section{Additional Document Sectionning Macros}

The package \texttt{upmethodology-fmt} provides several macros that permit to create special sections.

\subsection{Non-numbered Part in Table of Content}

If you want to add a document part that has no part number but appearing inside the table of content, the classical \LaTeX\ macros \texttt{{\textbackslash}part} and \texttt{{\textbackslash}part*} are inefficient. Indeed, \texttt{{\textbackslash}part} is adding a numbered part inside the table of content, and \texttt{{\textbackslash}part*} is adding an unnumbered part but not inside the table of content.

To add a unnumbered part inside the table of content, you could use one of the macros: \\
\texttt{{\textbackslash}parttoc[toctitle]\{title\}} \\
\texttt{{\textbackslash}parttoc*[toctitle]\{title\}}

The macros \texttt{{\textbackslash}parttoc} and \texttt{{\textbackslash}parttoc*} have the same effect except that \texttt{{\textbackslash}parttoc*} aligns the part's title to the other numbered parts' titles; and \texttt{{\textbackslash}parttoc} not.

\subsection{Non-numbered Chapter in Table of Content}

If you want to add a document chapter that has no chapter number but appearing inside the table of content, the classical \LaTeX\ macros \texttt{{\textbackslash}chapter} and \texttt{{\textbackslash}chapter*} are inefficient. Indeed, \texttt{{\textbackslash}chapter} is adding a numbered chapter inside the table of content, and \texttt{{\textbackslash}chapter*} is adding an unnumbered chapter but not inside the table of content.

To add a unnumbered chapter inside the table of content, you could use one of the macros: \\
\texttt{{\textbackslash}chaptertoc[toctitle]\{title\}} \\
\texttt{{\textbackslash}chaptertoc*[toctitle]\{title\}}

The macros \texttt{{\textbackslash}chaptertoc} and \texttt{{\textbackslash}chaptertoc*} have the same effect except that \texttt{{\textbackslash}chaptertoc*} aligns the chapter's title to the other numbered chapters' titles; and \texttt{{\textbackslash}chaptertoc} not.

\subsection{Non-numbered Section in Table of Content}

If you want to add a document section that has no a section number but appearing inside the table of content, the classical \LaTeX\ macros \texttt{{\textbackslash}section} and \texttt{{\textbackslash}section*} are inefficient. Indeed, \texttt{{\textbackslash}section} add a numbered section inside the table of content, and \texttt{{\textbackslash}section*} adds an unnumbered section but not inside the table of content.

To add a unnumbered section inside the table of content, you could use one of the macros: \\
\texttt{{\textbackslash}sectiontoc[toctitle]\{title\}} \\
\texttt{{\textbackslash}sectiontoc*[toctitle]\{title\}}

The macros \texttt{{\textbackslash}sectiontoc} and \texttt{{\textbackslash}sectiontoc*} have the same effect except that \texttt{{\textbackslash}sectiontoc*} aligns the section's title to the other numbered sections' titles; and \texttt{{\textbackslash}sectiontoc} not.

\subsection{Non-numbered Subsection in Table of Content}

If you want to add a document subsection that has no subsection number but appearing inside the table of content, the classical \LaTeX\ macros \texttt{{\textbackslash}subsection} and \texttt{{\textbackslash}subsection*} are inefficient. Indeed, \texttt{{\textbackslash}subsection} is adding a numbered subsection inside the table of content, and \texttt{{\textbackslash}subsection*} is adding an unnumbered subsection but not inside the table of content.

To add a unnumbered subsection inside the table of content, you could use one of the macros: \\
\texttt{{\textbackslash}subsectiontoc[toctitle]\{title\}} \\
\texttt{{\textbackslash}subsectiontoc*[toctitle]\{title\}}

The macros \texttt{{\textbackslash}subsectiontoc} and \texttt{{\textbackslash}subsectiontoc*} have the same effect except that \texttt{{\textbackslash}subsectiontoc*} aligns the subsection's title to the other numbered subsections' titles; and \texttt{{\textbackslash}subsectiontoc} not.

\subsection{Non-numbered Subsubsection in Table of Content}

If you want to add a document subsubsection that has no subsubsection number but appearing inside the table of content, the classical \LaTeX\ macros \texttt{{\textbackslash}subsubsection} and \texttt{{\textbackslash}subsubsection*} are inefficient. Indeed, \texttt{{\textbackslash}subsubsection} is adding a numbered subsubsection inside the table of content, and \texttt{{\textbackslash}subsubsection*} is adding an unnumbered subsubsection but not inside the table of content.

To add a unnumbered subsubsection inside the table of content, you could use one of the macros: \\
\texttt{{\textbackslash}subsubsectiontoc[toctitle]\{title\}} \\
\texttt{{\textbackslash}subsubsectiontoc*[toctitle]\{title\}}

The macros \texttt{{\textbackslash}subsubsectiontoc} and \texttt{{\textbackslash}subsubsectiontoc*} have the same effect except that \texttt{{\textbackslash}subsubsectiontoc*} aligns the subsubsection's title to the other numbered subsubsections' titles; and \texttt{{\textbackslash}subsubsectiontoc} not.


%###########################################################

\chapter{Package upmethodology-document}

\begin{center}
	\texttt{Version: \VERdoc}
\end{center}

The package \texttt{upmethodology-document} provides base function to manage document information (project, subproject, authors...).

\section{Document Information and Declaration}

The informations associated to an UP document are:
\begin{itemize}
\item \texttt{{\textbackslash}theupmproject} is the name of the project for which the document was produced;
\item \texttt{{\textbackslash}theupmsubproject} is the name of the sub-project for which the document was produced;
\item \texttt{{\textbackslash}theupmdocname} is the name of the document;
\item \texttt{{\textbackslash}theupmdocref} is the reference number of the document;
\item \texttt{{\textbackslash}theupmfulldocname} is the complete name of the document (composing by the project, subp-project and name of the document).
\end{itemize}

You could declare the information about your document with one of the following functions: \\
\texttt{{\textbackslash}declaredocument\{project\}\{name\}\{ref\}} \\
\texttt{{\textbackslash}declaredocumentex\{project\}\{subproject\}\{name\}\{ref\}} \\
where the parameters are:
\begin{itemize}
\item \texttt{project} is the name of the project for which the document is for;
\item \texttt{subproject} is the name of the sub-project for which the document is for;
\item \texttt{name} is the name of the document;
\item \texttt{ref} is the reference number of the document.
\end{itemize}

\section{Abstract and Key-words}

You are able to declare the abstract and the key-words for your document. Both are basically used by the back page package.

\subsection{Declarations}

The macro \texttt{{\textbackslash}setdocabstract} is for entering the docment's abstract:\\
\texttt{{\textbackslash}setdocabstract{[}lang{]}\{abstract\_text\}} \\
where \texttt{abstract\_text} is the text of your abstract and \texttt{lang} designates for which language the abstract text is for. If the language is not specified, this macro uses the current document language.

The macro \texttt{{\textbackslash}setdockeywords} is for entering the docment's key-words:\\
\texttt{{\textbackslash}setdockeywords{[}lang{]}\{keywords\}} \\
where \texttt{keywords} is the list of key-words and \texttt{lang} designates for which language the key-words are for. If the language is not specified, this macro uses the current document language.

\subsection{Rendering}

The macro \texttt{{\textbackslash}theupmdocabstract} is expanded with the abstract text:\\
\texttt{{\textbackslash}theupmdocabstract}

The macro \texttt{{\textbackslash}theupmdockeywords} is expanded with the key-words:\\
\texttt{{\textbackslash}theupmdockeywords}

\section{Document Summary}

You can obtain a document summary with the macro \texttt{{\textbackslash}upmdocumentsummary{[}width{]}} which produces:

\upmdocumentsummary

\section{Change Icons}

By default, this package uses the logo of \arakhneorg as icons. You could change them with the macros:

\begin{itemize}
\item \texttt{{\textbackslash}defupmsmalllogo\{filename\}} defines the small logo used in the headers for instance;
\item \texttt{{\textbackslash}defupmlogo\{filename\}} defines the logo used on the front page for instance.
\end{itemize}

The logos' filenames are accessible with the functions \texttt{{\textbackslash}theupmsmalldoclogo} and \texttt{{\textbackslash}theupmdoclogo}.

\section{Document Authors}

An author is someone who participates to the writing of the document. You could register author identities with: \\
\texttt{{\textbackslash}addauthor[email]\{firstname\}\{name\}} \\
\texttt{{\textbackslash}addauthor*[email]\{firstname\}\{name\}\{comment\}} \\
\texttt{{\textbackslash}addauthorvalidator[email]\{firstname\}\{name\}} \\
\texttt{{\textbackslash}addauthorvalidator*[email]\{firstname\}\{name\}\{comment\}}

The list of the authors is accessible by two means:
\begin{itemize}
\item \texttt{{\textbackslash}theauthorlist} is a coma-separated list of the authors' names;
\item \texttt{{\textbackslash}upmdocumentauthors} procudes an array of all the authors (see below for an example).
\end{itemize}

\upmdocumentauthors

\section{Document Validators}

A validator is someone who participates to the validation of the document. You could register validator identities with: \\
\texttt{{\textbackslash}addvalidator[email]\{firstname\}\{name\}} \\
\texttt{{\textbackslash}addvalidator*[email]\{firstname\}\{name\}\{comment\}} \\
\texttt{{\textbackslash}addauthorvalidator[email]\{firstname\}\{name\}} \\
\texttt{{\textbackslash}addauthorvalidator*[email]\{firstname\}\{name\}\{comment\}}

The list of the validators is accessible by two means:
\begin{itemize}
\item \texttt{{\textbackslash}thevalidatorlist} is a coma-separated list of the validator' names;
\item \texttt{{\textbackslash}upmdocumentvalidators} procudes an array of all the validators (see below for an example).
\end{itemize}

\upmdocumentvalidators

\section{Informed People}

An informed people is someone who receives the document to be informed about its content. You could register informed people identities with: \\
\texttt{{\textbackslash}addinformed[email]\{firstname\}\{name\}} \\
\texttt{{\textbackslash}addinformed*[email]\{firstname\}\{name\}\{comment\}}

The list of the informed people is accessible by two means:
\begin{itemize}
\item \texttt{{\textbackslash}theinformedlist} is a coma-separated list of the informed people' names;
\item \texttt{{\textbackslash}upmdocumentinformedpeople} procudes an array of all the informed people (see below for an example).
\end{itemize}

\upmdocumentinformedpeople

\section{Copyright and Publication Information}

Package \texttt{upmethodology-document} provides several macros to define the copyright owner and the publication informations required to generate a publication page.

\subsection{Setting Information}

Copyrighter is the people or the institution, or both, which is owning the copyright on the document. The following macro permits to set the identify of the copyighter in all the parts of the documents: \\
\texttt{{\textbackslash}setcopyrighter\{name\}}\\

Publisher is the people or the institution, or both, which is publishing the document. Basically it is the same the copyrighter (see above): \\
\texttt{{\textbackslash}setpublisher\{name\}}\\

Some times, copyright laws depend on the location where the document is printed. The following macro permits to put a message in the publication page which is indicating where the document is printed: \\
\texttt{{\textbackslash}setprintingaddress\{address\}}\\

Publications may be identifier by international identifiers. Package \texttt{upmethodology-document} supports ISBN, ISSN and DOI:
\texttt{{\textbackslash}setisbn\{number\}}\\
\texttt{{\textbackslash}setissn\{number\}}\\
\texttt{{\textbackslash}setdoi\{number\}}\\

\subsection{Retreiving Information}

The information set by the macros described in the previous section may be retreived with the following macros: \\
\texttt{{\textbackslash}theupmcopyrighter}\\
\texttt{{\textbackslash}theupmpublisher}\\
\texttt{{\textbackslash}theupmprintedin}\\
\texttt{{\textbackslash}theupmisbn}\\
\texttt{{\textbackslash}theupmissn}\\
\texttt{{\textbackslash}theupmdoi}\\

\subsection{Publication Page}

The package \texttt{upmethodology-document} provides the \texttt{{\textbackslash}upmpublicationpage} macro which is displaying a empty page with publication informations and optionally set the page number (default value is $-1$). Figure~\ref{fig:publication:page} illustrates the publication page of this document.

\begin{figure}
	\begin{center}
		\begin{framedminipage}{.8\linewidth}
		\upmpublicationminipage % in place of \upmpublicationpage to avod page number and page style changes
		\end{framedminipage}
		\caption{Example of Publication Page generated with \texttt{{\textbackslash}upmpublicationpage}}
		\label{fig:publication:page}
	\end{center}
\end{figure}

\section{Localization}

The following macros defines some localized strings used by \texttt{upmethodology-document}:
\begin{itemize}
\item \texttt{{\textbackslash}upm@lang@project}: Project;
\item \texttt{{\textbackslash}upm@lang@document}: Document;
\item \texttt{{\textbackslash}upm@lang@docref}: Reference;
\item \texttt{{\textbackslash}upm@lang@lastupdate}: Last Update;
\item \texttt{{\textbackslash}upm@lang@document@summary}: Document Summary;
\item \texttt{{\textbackslash}upm@lang@document@authors}: Authors;
\item \texttt{{\textbackslash}upm@lang@document@validators}: Validators;
\item \texttt{{\textbackslash}upm@lang@document@names}: Names;
\item \texttt{{\textbackslash}upm@lang@document@emails}: Emails;
\item \texttt{{\textbackslash}upm@lang@document@initials}: Initials;
\item \texttt{{\textbackslash}upm@lang@document@abstract}: Abstract;
\item \texttt{{\textbackslash}upm@lang@document@keywords}: Key-words.
\end{itemize}


%###########################################################

\chapter{Package upmethodology-frontpage}

\begin{center}
	\texttt{Version: \VERfp}
\end{center}

The \texttt{upmethodology-frontpage} package provides an front page for the UP documents. This package does not provides any public function. It is based on all the previous packages.

\section{Display the front page}

The front cover is displayed by invoking one of the following macros: \\
\texttt{{\textbackslash}maketitle} \\
\texttt{{\textbackslash}makefrontcover} \\

\section{Change Front Page Layout}

It is possible to change the layout of the front page with the macro: \\
\texttt{{\textbackslash}setfrontlayout\{layout\_name\}}\\
where \texttt{layout\_name} must be one of:
\begin{itemize}
\item \texttt{classic}: classic front page layout with title and logo;
\item \texttt{modern}: front page layout with title and logo and background picture.
\end{itemize}

The figure~\figref{frontpage:layout} illustrates the differents layouts.

\begin{mfigures}{Front Page Layouts}{frontpage:layout}
	\mbox{}\hfill
	\msubfigure{width=.45\linewidth}{frontclassic}{\texttt{classic}}
	\hfill
	\msubfigure{width=.45\linewidth}{frontmodern}{\texttt{modern}}
	\hfill\mbox{}
\end{mfigures}

\section{Change Illustration Picture}

It is possible to insert an illustration picture on the front page. You could specify the image with the macro: \\
\texttt{{\textbackslash}setfrontillustration[width\_factor]\{filename\}} \\
where:
\begin{itemize}
\item \texttt{width\_factor} is the scaling factor of the picture according to the line width. If you specifies \texttt{1} the image will not be scaled, for \texttt{.5} the image will be the half of its original width...
\item \texttt{filename} is the name of picture to use as the illustration.
\end{itemize}

\section{Define a Front Page in Extensions}

The \texttt{upmethodology-frontpage} package is able to use a page layout defined in a document extension (see chapter~\ref{section:document:extension} for details on document extension).

\pagebreak A \LaTeX\ macro must be defined in the \texttt{upmext-NAME.cfg} file of the extension. The name of this macro (for example \texttt{mylayout}) must be set with the \texttt{{\textbackslash}set} macro in the same file:\\
\texttt{{\textbackslash}set\{frontpage\}\{mylayout\}}\\

\section{Localization}

The following macros defines some localized strings used by \texttt{upmethodology-frontpage}:
\begin{itemize}
\item \texttt{{\textbackslash}upm@lang@front@authors}: Authors;
\end{itemize}

%###########################################################

\chapter{Package upmethodology-backpage}

\begin{center}
	\texttt{Version: \VERbp}
\end{center}

The package \texttt{upmethodology-backpage} provides an back page for the UP documents. This package does not provides any public function. It is based on all the previous packages.

\section{Display the back page}

The back cover is displayed by invoking the following macro: \\
\texttt{{\textbackslash}makebackcover} \\

\section{Change Back Page Layout}

It is possible to change the layout of the back page with the macro: \\
\texttt{{\textbackslash}setbacklayout\{layout\_name\}}\\
where \texttt{layout\_name} must be one of:
\begin{itemize}
\item \texttt{none}: no back page.
\end{itemize}

\section{Define a Back Page in Extensions}

The \texttt{upmethodology-backpage} package is able to use a page layout defined in a document extension (see chapter~\ref{section:document:extension} for details on document extension).

A \LaTeX\ macro must be defined in the \texttt{upmext-NAME.cfg} file of the extension. The name of this macro (for example \texttt{mylayout}) must be set with the \texttt{{\textbackslash}set} macro in the same file:\\
\texttt{{\textbackslash}set\{backpage\}\{mylayout\}}\\


%###########################################################

\chapter{Package upmethodology-extension}\label{section:document:extension}

\begin{center}
	\texttt{Version: \VERext}
\end{center}

The package \texttt{upmethodology-extension} provides tools to create layout and rendering extensions.
It is possible to write an extension to the \texttt{upmethodology-document} package. An extension is able to override several values from the default \texttt{upmethodology-}packages or may be used by the other suite's packages. For example, the Systems and Transport laboratory\reffootnote*{foot:setlab} extension is providing laboratory's icons, publisher's name and page layouts.

\section{Load a Document Extension}

To load and use a document extension, you must invoke the macro:\\
\texttt{{\textbackslash}UseExtension\{extension\_name\}}\\
where \texttt{extension\_name} is the identifier of the extension to load. The extension's files must be inside your \LaTeX\ search path.

\section{Write a Document Extension}

A document extension could be written and described inside a file named \texttt{upmext-NAME.cfg}, where \texttt{NAME} is the name of the extension. This file must be put in your \LaTeX\ search path.

The \texttt{upmext-NAME.cfg} file is a \LaTeX\ file in which a set of definition macros are put. These macros must respect the \LaTeX\ syntax.

The \texttt{{\textbackslash}DeclaraCopyright} macro permits to declare several copyright information about the extension: \\
\texttt{{\textbackslash}DeclareCopyright{[}lang{]}\{extension\_name\}\{year\}\{copyrighter\}\{trademark and copyright information\}} \\

This macro declares the \texttt{copyright} value which contains the copyright text (for this documentation ``\Get{copyright}'').
This macro also declares the \texttt{trademarks} value which contains the trademark and other related informations about the extension (for this documentation ``\Get{trademarks}'').

Additional macros are provided to redefine the \texttt{upmethodology-document} constants:\\
\texttt{{\textbackslash}Set{[}lang{]}\{variable\_name\}\{value\}}\\

The \texttt{variable\_name} is the name of the value to override. It must be taken in one of the names listed in table~\tabref{documentextension:names:set}. The \texttt{lang} parameter is a language identifier. It is used to restrict the definition to a specific language. If not given, the default language is used insteed. The \texttt{image\_name} and \texttt{image\_scale} are the name of the image file and the scaling factor respectively.

\begin{mtable}{\linewidth}{2}{lX}{List of overiddable value names}{documentextension:names:set}
	\tabularheader{Value Name}{Description}
	\hline
	logo & the filename of the picutre which must be used as a large logo. \\
	\hline
	smalllogo & the filename of the picutre which must be used as a small logo. \\
	\hline
	copyrighter & the name of the authors or the institution which own the copyright on the document. \\
	\hline
	publisher & the name of the document's publisher. The \texttt{lang} parameter is supported. \\
	\hline
	printedin & the location/address where this document is printed. \\
	\hline
	frontillustration & the image to use as illustration. The \texttt{lang} parameter is ignored. \\
	\hline
	frontpage & the name of the front page style --- not the \LaTeX\ macros --- to layout the front page. \newline OR \newline the front page illustration.\\
	\hline
	backpage & the \LaTeX\ macros to layout the back page.  \newline OR \newline the back page illustration.\\
	\hline
	cfrontpage & the \LaTeX\ macros --- not the name of the front page style --- to layout the front page.\\
	\hline
\end{mtable}

The \texttt{{\textbackslash}Get} macro permits to retreive the value defined by a \texttt{{\textbackslash}Set}:\\
\texttt{{\textbackslash}Get\{variable\_name\}} \\

The \texttt{{\textbackslash}Unset} macro permits to remove the definition of a value:\\
\texttt{{\textbackslash}Unset\{variable\_name\}} \\

The \texttt{{\textbackslash}Ifnotempty} macro permits to expand the \LaTeX\ macros if the given text is not empty:\\
\texttt{{\textbackslash}Ifnotempty\{text\}\{latex\_code\}} \\

The \texttt{{\textbackslash}Ifempty} macro permits to expand the \LaTeX\ macros if the given text is empty:\\
\texttt{{\textbackslash}Ifempty\{text\}\{latex\_code\}} \\

The \texttt{{\textbackslash}Ifelsedefined} macro permits to expand the \LaTeX\ macros in \texttt{then\_code} if a value with the given name was defined, or to expand the \LaTeX\ macros in \texttt{else\_code} if no value with the given name was defined:\\
\texttt{{\textbackslash}Ifelsedefined\{value\_name\}\{then\_code\}\{else\_code\}} \\

The \texttt{{\textbackslash}Put} macro is an extension of the standard picture \texttt{{\textbackslash}put} macro. It takes into account the joint margin applied in two sided documents when it is used on page's backside (eg. the back page of the document):\\
\texttt{{\textbackslash}Put(x,y)\{macros\}} \\

This macro must be used inside a \texttt{picture} environment in place of the standard \texttt{{\textbackslash}put} macro.

%###########################################################

\chapter{Package upmethodology-task}

\begin{center}
	\texttt{Version: \VERtask}
\end{center}

The \LaTeX\ package \texttt{upmethodology-task} provides a set of macros to define project's tasks.

During \LaTeX\ compilation this package could log the message \texttt{"Project Task(s) may have changed. Rerun to get cross-references right"} when some task information was not found or due to cross-references on them.

\section{Task Definition}

The definition of a task could be made only inside one of the following environments: \\
\texttt{{\textbackslash}begin\{taskdescription\}\{id\}...{\textbackslash}end\{taskdescription\}} \\
\texttt{{\textbackslash}begin\{taskdescription*\}\{id\}...{\textbackslash}end\{taskdescription*\}} \\
where \texttt{id} is the identifier of the task.

The environment \texttt{taskdefinion} displays the task's description with a call to \texttt{{\textbackslash}thetaskdescription\{id\}}. In the opposite \texttt{taskdefinition*} never displays the ta's description.

Inside one of the task's definition environment above, you could use one of the following macros to define the task's attributes:
\begin{itemize}
\item \texttt{{\textbackslash}taskname\{name\}} \\
	permits to defines the name of the task;
\item \texttt{{\textbackslash}tasksuper\{id\}} \\
	indicates that the current task is a sub-task of the task identified by the given identifier;
\item \texttt{{\textbackslash}taskcomment\{text\}} \\
	permits to describe the task's purposes and goals (will be shown in the description box of the task's description);
\item \texttt{{\textbackslash}taskprogress\{percent\}} \\
	defines the percent for thtask's archieving;
\item \texttt{{\textbackslash}taskstart\{date\}} \\
	permits to set the starting date of the task (real or predicted);
\item \texttt{{\textbackslash}taskend\{date\}} \\
	permits to set the finished date of the task (real or predicted);
\item \texttt{{\textbackslash}taskmanager\{name\}} \\
	adds a task's manager into the list of the managers;
\item \texttt{{\textbackslash}taskmember\{name\}} \\
	adds a task's member into the list of the members;
\item \texttt{{\textbackslash}taskmilestone\{date\}\{comment\}} \\
	add a milestone into the task for the given date and described by the given comment.
\end{itemize}

\section{Task Reference}

You could reference any information about the defined tasks in your document. In case you used cross-references this package could log the message "\verb+Project Task(s) may have changed. Rerun to get cross-references right+" to complain about rebuilding of our document.

The following macros are available:
\begin{itemize}
\item \texttt{{\textbackslash}thetasksuper\{id\}} \\
	replies the identifier of the parent task corresponding to the task identified by \texttt{id};
\item \texttt{{\textbackslash}thetaskname\{id\}} \\
	replies the name of the the task identified by \texttt{id};
\item \texttt{{\textbackslash}thetaskcomment\{id\}} \\
	replies the description for the the task identified by \texttt{id};
\item \texttt{{\textbackslash}thetaskprogress\{id\}} \\
	replies the archieving percent for the the task identified by \texttt{id};
\item \texttt{{\textbackslash}thetaskstart\{id\}} \\
	replies the starting date for the the task identified by \texttt{id};
\item \texttt{{\textbackslash}thetaskend\{id\}} \\
	replies the ending date for the the task identified by \texttt{id};
\item \texttt{{\textbackslash}thetaskmanagers\{id\}} \\
	replies the managers' list for the the task identified by \texttt{id};
\item \texttt{{\textbackslash}thetaskmembers\{id\}} \\
	replies the members' list for the the task identified by \texttt{id};
\item \texttt{{\textbackslash}thetaskmilestones\{id\}} \\
	replies the list of milestone's dates for the the task identified by \texttt{id};
\item \texttt{{\textbackslash}thetaskmilestonecomment\{id\}\{date\}} \\
	replies the comment of the given milestone for the the task identified by \texttt{id};
\item \texttt{{\textbackslash}thetaskdescription[width]\{id\}} \\
	replies the complete description of the the task identified by \texttt{id}.
\end{itemize}

\section{Localization}

The following macros defines some localized strings used by \texttt{upmethodology-task}:
\begin{itemize}
\item \texttt{{\textbackslash}upm@task@lang@task}: Task;
\item \texttt{{\textbackslash}upm@task@lang@escription}: Description;
\item \texttt{{\textbackslash}upm@task@lang@startat}: Start at;
\item \texttt{{\textbackslash}upm@task@lang@endat}: End at;
\item \texttt{{\textbackslash}upm@task@lang@archieved}: Archieved;
\item \texttt{{\textbackslash}upm@task@lang@managers}: Managers;
\item \texttt{{\textbackslash}upm@task@lang@members}: Members;
\item \texttt{{\textbackslash}upm@task@lang@Milestones}: Milestones;
\item \texttt{{\textbackslash}upm@task@lang@subtask}: Sub-task of.
\end{itemize}

%###########################################################

\chapter{Package upmethodology-code}

\begin{center}
	\texttt{Version: \VERcode}
\end{center}

The \LaTeX\ package \texttt{upmethodology-code} provides a set of macros for source code formatting. The supported source codes are UML, Java and C++.

You could load the package with the following options:
\begin{center}
\begin{tabular}{|>{\ttfamily}l|l|}
\hline
uml & use the UML notation (default value)\\
java & use the Java notation \\
cpp & use the C++ notation \\
\hline
\end{tabular}
\end{center}

You could also change the notation language with the macro: \\
\texttt{{\textbackslash}upmcodelang\{upm\string|java\string|cpp\}}

The provided macros are listed in the following table: \\
\begin{tabularx}{\linewidth}{|>{\ttfamily}l|X|X|X|}
	\hline
	{\normalfont macro} & UML & Java & C++ \\
	\hline
	\multicolumn{4}{|X|}{Prototypes} \\
	\hline
	{\textbackslash}jclass\{TheClass\} & \upmcodelang{uml}\jclass{TheClass} & \upmcodelang{java}\jclass{TheClass} & \upmcodelang{cpp}\jclass{TheClass} \\
	{\textbackslash}jinterface\{TheInterface\} & \upmcodelang{uml}\jinterface{TheInterface} & \upmcodelang{java}\jinterface{TheInterface} & \upmcodelang{cpp}\jinterface{TheInterface} \\
	{\textbackslash}jpackage\{ThePackage\} & \upmcodelang{uml}\jpackage{ThePackage} & \upmcodelang{java}\jpackage{ThePackage} & \upmcodelang{cpp}\jpackage{ThePackage} \\
	{\textbackslash}jfunc\{FunctionName\} & \upmcodelang{uml}\jfunc{FunctionName} & \upmcodelang{java}\jfunc{FunctionName} & \upmcodelang{cpp}\jfunc{FunctionName} \\
	\hline
	\multicolumn{4}{|X|}{Types} \\
	\hline
	{\textbackslash}jclazz & \upmcodelang{uml}\jclazz & \upmcodelang{java}\jclazz & \upmcodelang{cpp}\jclazz \\
	{\textbackslash}jvoid & \upmcodelang{uml}\jvoid & \upmcodelang{java}\jvoid & \upmcodelang{cpp}\jvoid \\
	{\textbackslash}jboolean & \upmcodelang{uml}\jboolean & \upmcodelang{java}\jboolean & \upmcodelang{cpp}\jboolean \\
	{\textbackslash}jint & \upmcodelang{uml}\jint & \upmcodelang{java}\jint & \upmcodelang{cpp}\jint \\
	{\textbackslash}jlong & \upmcodelang{uml}\jlong & \upmcodelang{java}\jlong & \upmcodelang{cpp}\jlong \\
	{\textbackslash}jfloat & \upmcodelang{uml}\jfloat & \upmcodelang{java}\jfloat & \upmcodelang{cpp}\jfloat \\
	{\textbackslash}jdouble & \upmcodelang{uml}\jdouble & \upmcodelang{java}\jdouble & \upmcodelang{cpp}\jdouble \\
	{\textbackslash}jchar & \upmcodelang{uml}\jchar & \upmcodelang{java}\jchar & \upmcodelang{cpp}\jchar \\
	{\textbackslash}jstring & \upmcodelang{uml}\jstring & \upmcodelang{java}\jstring & \upmcodelang{cpp}\jstring \\
	{\textbackslash}jarray\{T\} & \upmcodelang{uml}\jarray{\jclass{T}} & \upmcodelang{java}\jarray{\jclass{T}} & \upmcodelang{cpp}\jarray{\jclass{T}} \\
	{\textbackslash}jcollection\{T\} & \upmcodelang{uml}\jcollection{\jclass{T}} & \upmcodelang{java}\jcollection{\jclass{T}} & \upmcodelang{cpp}\jcollection{\jclass{T}} \\
	{\textbackslash}jset\{T\} & \upmcodelang{uml}\jset{\jclass{T}} & \upmcodelang{java}\jset{\jclass{T}} & \upmcodelang{cpp}\jset{\jclass{T}} \\
	\hline
\end{tabularx}
\newpage
\begin{tabularx}{\linewidth}{|>{\ttfamily}l|X|X|X|}
	\hline
	{\normalfont macro} & UML & Java & C++ \\
	\hline
	\multicolumn{4}{|X|}{Constants} \\
	\hline
	{\textbackslash}jtrue & \upmcodelang{uml}\jtrue & \upmcodelang{java}\jtrue & \upmcodelang{cpp}\jtrue \\
	{\textbackslash}jfalse & \upmcodelang{uml}\jfalse & \upmcodelang{java}\jfalse & \upmcodelang{cpp}\jfalse \\
	\hline
	\multicolumn{4}{|X|}{Operations} \\
	\hline
	{\textbackslash}jcode\{source code\} & \upmcodelang{uml}\jcode{source code} & \upmcodelang{java}\jcode{source code} & \upmcodelang{cpp}\jcode{source code} \\
	{\textbackslash}jcall\{fct\}\{params\} & \upmcodelang{uml}\jcall{fct}{params} & \upmcodelang{java}\jcall{fct}{params} & \upmcodelang{cpp}\jcall{fct}{params} \\
	{\textbackslash}jop\{operator\} & \upmcodelang{uml}\jop{operator} & \upmcodelang{java}\jop{operator} & \upmcodelang{cpp}\jop{operator} \\
	\hline
\end{tabularx}

%###########################################################

\chapter{Authors and License}

Copyright \copyright\ \upmcopyrightdate\ \makename{St\'ephane}{Galland}

\vspace{.5cm}

This program is free library; you can redistribute it and/or modify it under the terms of the GNU Lesser General Public License as published by the Free Software Foundation; either version 3 of the License, or any later version.

\vspace{.5cm}

This library is distributed in the hope that it will be useful, but WITHOUT ANY WARRANTY; without even the implied warranty of MERCHANTABILITY or FITNESS FOR A PARTICULAR PURPOSE.  See the GNU Lesser General Public License for more details.

\vspace{.5cm}

You should have received a copy of the GNU Lesser General Public License along with this library; see the file COPYING.  If not, write to the Free Software Foundation, Inc., 59 Temple Place - Suite 330,
Boston, MA 02111-1307, USA.

\end{document}
